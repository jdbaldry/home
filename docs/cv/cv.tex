%!TEX encoding = UTF-8 Unicode
% LaTeX CV

\documentclass{cv}
\usepackage{tabularx}
% Simpler list syntax and infinite nesting.
\usepackage[ampersand]{easylist}
% Don't want numbering of subsections.
\setcounter{secnumdepth}{0}

% Quick trick to remove footer without using fancyhdr package.
\makeatletter
\let\@oddfoot\@empty
\let\@evenfoot\@empty
\makeatother

% For debugging.
%\usepackage{showframe}

\begin{document}
{\small\begin{tabular}{l}
  github.com/jdbaldry \\
  linkedin.com/in/jackbaldry
\end{tabular}}\hfill%
{\LARGE\bfseries\begin{tabular}{l}
  JACK BALDRY
\end{tabular}}\hfill%
{\small\begin{tabular}{r}
  +44 7410 114 999 \\
  cv@jdb.sh
\end{tabular}}%

\section{Summary}
Software engineer with a strong operational background.
Experienced with administrating Kubernetes and developing distributed systems deployed upon it.
Keenly interested in understanding technologies and sharing knowledge.

\section{Work}
\datedsubsection{May 2019 -- present}{Senior Software Engineer at Grafana Labs Ltd.}

At Grafana, I work as a software engineer developing our horizontally scalable, long-term storage time series database projects; Cortex and Grafana Enterprise Metrics.

I contribute to all aspects of development including feature development, tooling, documentation, CI/CD, configuration management, and infrastructure as code.

I kickstarted development of a Jsonnet language server that is used by many engineers within the company to improve effectiveness workin with Jsonnet.

\datedsubsection{Mar 2017 -- May 2019}{Principal DevOps Engineer at Oracle Data Cloud}

During my time at Oracle Data Cloud (formerly Grapeshot Ltd.) I have progressed from being a Junior System Administrator of Red Hat based systems to a Principal DevOps Engineer. I am the primary maintainer of multiple bespoke Kubernetes clusters built by myself and Tom Wilkie on AWS. Alongside supporting the platform, I provide support and architectural guidance for the twenty plus developer customers and deploy my own tooling upon it.

\subsubsection{Kubernetes}
The design provides maximum control over the Kubernetes control plane with a focus on Mean Time to Recovery. We encourage stateless deployments using external cloud vendor neutral storage services to simplify complexity while avoiding vendor lock-in. To provide a highly available cluster, I operate three Etcd clusters, one for the Kubernetes events, another for the main store and a third as a backend store for a Vault cluster. 

Key tooling:

\begin{easylist}[itemize]
& Terraform is used to manage the AWS infrastructure as code.

& Puppet is responsible for Linux level configuration management.

& Tanka is used by my CI/CD pipelines to deploy to Kubernetes.

& Jsonnet is used for templating Kubernetes manifests as well as Terraform modules.

\end{easylist}

\subsubsection{Monitoring}
I lead the design and deployment of improved monitoring infrastructure to reduce noise, highlight actionable alerts, and enforce SLAs with a focus of remaining manageable at scale.

\begin{easylist}[itemize]
I have deployed monitoring tooling to facilitate all three pillars of observability:

& Fluentd collects and filters Kubernetes logs and ships them to Elasticsearch.

& Jaeger is deployed with a centralized Elasticsearch data store to provide distributed tracing throughout the Kubernetes and bare metal infrastructure.

& Prometheus and Alertmanager are used for short term (30d) metrics monitoring and alerting. I presented a proof of concept deployment of Cortex for multi-tenant Prometheus metrics and long term storage. I consume and have contributed to a number of open-source monitoring mixins.

& Grafana dashboards are used to graph various data sources including Prometheus, MySQL and Elasticsearch with Jsonnet libraries used to simplify creation and facilitate version control. I have contributed the grafonnet-lib heatmap panel that use internally. Dashboards principally follow the USE or RED method where appropriate but I also ensure that panels and dashboards have a hierarchy for useful overview and drill down.
\end{easylist}

\subsubsection{Cloud}
I manage extensive deployments in AWS using Terraform, primarily using EC2, RDS, Elasticsearch Service and Elasticache. My most recent work project is informing the migration of bare metal servers to Oracle Cloud Infrastructure (OCI) bare metal shapes using Terraform. I also have some working experience of Google Cloud through contributing to departureboard.io infrastructure that uses Firebase, GKE and Stackdriver.

\subsubsection{Documentation and knowledge sharing}
I practice and encourage knowledge sharing through a number of practical methods:

\begin{easylist}[itemize]
& I run weekly show and tell meetings for the infrastructure team to present and discuss practices and ideas.
& I write and encourage the writing of design documents and postmortems in full prose to ensure ideas are fully formed and clearly conveyed.
& I prioritize technical documentation for new projects and features. Recently, I worked through the \emph{Google Technical Writing Course} (https://github.com/jdbaldry/home/blob/master/docs/techinical-writing-course.md).
\end{easylist}

\subsubsection{Development}
Though my work is primarily operationally focused, I have experience programming in Golang and Python as well as exposure to Scheme, Haskell and C through personal learning. I regularly write bash scripts and Makefiles for automating operational tasks.

Golang projects include:

\begin{easylist}[itemize]
  & \textbf{vault-secrets-mutating-webhook} (Oracle Internal) Kubernetes mutating webhook that allows developers to populate Kubernetes secrets from Vault KV store.
  & \textbf{tfsonnet} (https://github.com/jdbaldry/tfsonnet) is my personal MVP of a Terraform Jsonnet library generated from the Terraform providers schema.
  & \textbf{budget-grpc} (https://github.com/jdbaldry/budget-grpc) an exploration into GRPC, grpc-web and React.
\end{easylist}

\section{Work cont.}
\datedsubsection{Nov 2013 -- Mar 2017}{System Services Representative at IBM}

At IBM I spent my time working with a team of engineers to provide 24x7 hardware support for a large UK banking organization.
Support included IBM DS storage, System x, POWER systems and multi-vendor x86 support.
I focused on building and maintaining strong client relationships at all levels of engagement with all customers.
As I wasn't sure a career in hardware would suit me, I took IBM Giveback projects including the system administration of an internal buddying tool as well as maintaining and extending a LAMP based call management tool.

\section{Skills and Certification}
% Increasing table row spacing.
\renewcommand{\arraystretch}{1.5}
\newcolumntype{L}{>{\raggedright\arraybackslash}X}%
\begin{tabularx}{\textwidth}{ r|L }
  Infrastructure & Certified Kubernetes Administrator, Red Hat Certified Systems Administrator, working knowledge of Amazon Web Services, Google Cloud Platform, and limited experience with Azure. \\
  Configuration management / orchestration & Active experience managing Kubernetes manifests with Helm and Jsonnet. Previous experience with Ansible, Puppet, Terraform. \\
  Languages & Golang, Python, Bash, Jsonnet, Nix expression language. \\
  CI/CD & Active experience with GitHub Actions and Drone CI. Previous experience with Gitlab runners and Jenkins. \\
  Developer Experience (DX) & Confident with Make and Bash scripting for developer experience.
\end{tabularx}

\section{Education}
A2 levels in Mathematics, Further Mathematics, and Chemistry. AS Level in Biology. 12 GCSEs including English, Mathematics and Science.

\section{Interests}
I play five-a-side football thrice weekly and run during rest days.
I love to walk outdoors, travel, and cook.
I rarely turn down an opportunity to go to the pub.
\end{document}
